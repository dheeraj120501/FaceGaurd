\section{Requirement Analysis}

\subsection{Software \& Hardware Requirements}
\begin{frame}{Software \& Hardware Requirements}
	\begin{multicols}{2}
		Software Requirements:
		\begin{itemize}
			\item \textbf{Libraries \& frameworks:} PyTorch, LambdaCloud, Google Colab, ONNX Runtime, Express.
			\item \textbf{Programming languages:} Python, HTML, JS, CSS.
			\item \textbf{Database:} Faiss, Milvus, Annoy.
		\end{itemize}

		\break

		Hardware Requirements:
		\begin{itemize}
			\item \textbf{CPU:} Multi-core 2.5 GHz.
			\item \textbf{GPU:} 8GB VRAM for training.
			\item Stable internet connectivity.
			\item CCTVs.
		\end{itemize}
	\end{multicols}
	Link: \textcolor{blue}{\underline{\urllink{SRS.pdf}{System Requirements Specification}}}
\end{frame}

\subsection{Information Domain Analysis}
\begin{frame}{Information Domain Analysis}
	The system should:
	\begin{enumerate}
		\item Define how data is collected in real-time. Consider the methods used to capture facial images, including camera systems, body-worn cameras, CCTV cameras, or any other data sources. Determine the frequency and volume of data generated.
		\item Explore where and how the collected data is stored. This can involve databases, cloud storage, or local storage systems. Consider data security and privacy measures, especially in a law enforcement context.
		\item In the context of facial recognition, labeling data with identity information is vital. Determine how face images are annotated and labeled for training and identification purposes.
		\item Establish procedures for ensuring data quality, including error detection and correction. Consider how data is validated and cleaned.
	\end{enumerate}
\end{frame}

\subsection{External \& Internal Interfacing}
\begin{frame}{External \& Internal Interfacing}
	\begin{multicols}{2}
		External Interfacing:
		\begin{itemize}
			\item Primary user interface will be web-based, accessible via standard web browsers.
			\item The face recognition system needs to interface with external databases containing records of individuals for identification. 
			\item Law enforcement relies on various cameras and sensors, including body-worn cameras, CCTV cameras, and thermal imaging devices.
		\end{itemize}

		\break

		Internal Interfacing:
		\begin{itemize}
			\item The face detection module within the system needs to interface with other components. 
			\item Face tracking module interfaces with the detection module to keep track of faces as they move within the camera's field of view.
			\item The feature extraction and face encoding module interfaces with the tracked faces, extracting unique facial features and face embeddings.
		\end{itemize}
	\end{multicols}
\end{frame}

\subsection{Demand}
\begin{frame}{Demand}
	\begin{enumerate}
		\item The primary demand comes from law enforcement agencies, such as police departments, border control, and security organizations.
		\item The general public demands robust safety and security measures. They rely on law enforcement to protect their communities and respond promptly to threats.
		\item Government bodies and regulatory agencies often play a role in defining the standards and guidelines for the use of facial recognition technology in law enforcement.
		\item During crisis situations, such as natural disasters or public emergencies, there is a demand for rapid response and assistance. Real-time face recognition can help locate missing persons or identify individuals in need of assistance.
	\end{enumerate}
\end{frame}

\subsection{Stakeholders}
\begin{frame}{Stakeholders}
	\begin{enumerate}
		\item \textbf{Law Enforcement Agencies:} Police departments, federal agencies, and other law enforcement entities are primary stakeholders.
		\item \textbf{Government and Regulatory Bodies:} Government agencies responsible for law enforcement oversight and regulation are stakeholders who may provide guidance, standards, or compliance requirements.
		\item \textbf{Researchers and Developers:} Individuals or groups working on the development of face recognition algorithms, hardware, or software are stakeholders who aim to see their technologies effectively applied.
	\end{enumerate}
\end{frame}