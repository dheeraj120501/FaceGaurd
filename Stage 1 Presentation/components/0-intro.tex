\section{Introduction}

\subsection{Problem Statement}
\begin{frame}{Problem Statement}
	In law enforcement settings, effective real-time face recognition is paramount. Nevertheless, real-world challenges, including occlusions, crowded scenarios, and background disturbances, compromise existing systems. In this project we aim to design and prototype a real-time face recognition system overcoming the above challenges making it optimal for law enforcement applications.
\end{frame}

\subsection{Motivation}
\begin{frame}{Motivation}
	\begin{enumerate}
		\item Driving this research is the profound importance of real-time face recognition in modern-day law enforcement. 

		\item By surmounting the challenges posed by occlusions, crowds, and environmental disturbances, we aspire to elevate law enforcement capabilities, fostering quicker criminal identifications and enhancing overall public security. 
		
		\item The anticipated outcomes of this endeavour can reshape how law enforcement interacts with technology, offering both officers and the general populace a heightened sense of safety and assurance.	
	\end{enumerate}
\end{frame}

\subsection{Literature Review}
\begin{frame}[allowframebreaks]{Literature Review}
	A sample of the literature review is as follows:
	{
		\scriptsize
		\newlength{\limwidth}
		\setlength{\limwidth}{5.75cm}
		\begin{table}[htbp]
			\begin{center}
			\begin{tabularx}{\columnwidth}{|X|X|X|}
			\hline
			\textbf{Title} & \textbf{Advantages}& \textbf{Disadvantages} \\
			\hline
			GhostFaceNets: Lightweight Face Recognition Model From Cheap Operations & GhostFaceNets greatly improve efficiency for face verification tasks compared to previous SOTA mobile CNNs, making them suitable for deployment on devices with constrained memory and computational resources. & The evaluation of GhostFaceNets is limited to face verification tasks and does not include other biometric recognition tasks such as face identification or face clustering.
			\\
			\hline
			Towards Age-Invariant Face Recognition & The AIM model achieves continuous face rejuvenation/aging with remarkable photorealistic and identity-preserving properties, without the need for paired data or knowledge of the true age of testing samples. & the paper does not discuss the potential limitations or challenges of age-invariant face recognition in general, such as the impact of extreme age variations, changes in facial features due to aging, or the presence of occlusions or disguises.\\
			\hline
		\end{tabularx}
		\end{center}
		\end{table}			
	\pagebreak
		\begin{table}[htbp]
			\begin{center}
			\begin{tabularx}{\columnwidth}{|X|X|X|}
			\hline
			\textbf{Title} & \textbf{Advantages}& \textbf{Disadvantages} \\
			\hline
			Masked Face recognition Using Support Vector Machine and Convolutional Neural Network & The proposed masked face recognition algorithm, combining Convolutional Neural Network (CNN) and Support Vector Machine (SVM), has practical implications in improving the true acceptance rate in face image prediction. & The paper mentions the need for a customized training set that includes masked images, but it doesn't specify how large or diverse this dataset is. The performance of the proposed method could be limited by the dataset's size and representativeness.
			\\
			\hline
			DVG-Face: Dual Variational Generation for Heterogeneous Face Recognition & DVG-Face outperforms state-of-the-art methods on seven challenging databases belonging to five HFR tasks, including NIR-VIS, Sketch-Photo, Profile-Frontal Photo, Thermal-VIS, and ID-Camera. & The paper does not discuss the computational complexity or efficiency of the proposed DVG-Face framework.			\\
			\hline
		\end{tabularx}
		\end{center}
		\end{table}	
	\pagebreak	
		\begin{table}[htbp]
			\begin{center}
			\begin{tabularx}{\columnwidth}{|X|X|X|}
			\hline
			\textbf{Title} & \textbf{Advantages}& \textbf{Disadvantages} \\
			\hline
			Consistent Sub-Decision Network for Low-Quality Masked Face Recognition & The proposed consistent sub-decision network improves the performance of masked face recognition by focusing on the upper faces without occlusion and extracting more discriminative features. & The limitations of this paper are not explicitly mentioned in the provided sources.
			\\
			\hline
			Towards NIR-VIS Masked Face Recognition & The novel heterogeneous training method using semi-siamese networks maximizes the mutual information shared by the face representation of two domains, improving the performance of NIR-VIS masked face recognition. & Data Limitations,  Implementing 3D face reconstruction and semi-siamese networks can be computationally complex and may require specialized expertise and resources., Generalization to Other Modalities.		\\
			\hline
		\end{tabularx}
		\end{center}
		\end{table}		
		For detailed Literature review check the following: \textcolor{blue}{\underline{\urllink{Survey.pdf}{Survey Paper}}}
	}	
\end{frame}

\subsection{Research Gaps}
\begin{frame}{Research Gaps}
	From literature review it was evident that:
	\begin{enumerate}
		\item While numerous face recognition solutions exist for law enforcement purposes, there remains a conspicuous gap in their real-time applicability, particularly in challenging scenarios of occlusions, dense crowds, and dynamic backgrounds. 

		\item Current systems exhibit reduced accuracy and efficiency under these conditions, often leading to misidentifications or missed detections entirely. 
		
		\item The requirement for a real-time face recognition system, adept at navigating these specific challenges while maintaining swift and precise identification in on-the-ground law enforcement situations, has yet to be fully addressed in contemporary research.
		
	\end{enumerate}
\end{frame}

\subsection{Objectives}
\begin{frame}{Objectives}
	This project has the following objectives:
	\begin{enumerate}
		\item Design and prototype a real-time face recognition system optimised for law enforcement applications.
		\item Innovate algorithms to manage occlusions and crowded environments, ensuring uninterrupted face recognition capabilities.
		\item Conceptualise techniques to counteract the effects of changing backgrounds on recognition accuracy.
		\item Thoroughly validate the crafted system through meticulous real-world law enforcement tests.
	\end{enumerate}
\end{frame}