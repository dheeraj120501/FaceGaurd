\section{Introduction} \label{section:intro}
In the ever-evolving landscape of law enforcement, the integration of cutting-edge technology has become not just a convenience but a necessity. Among the technological advancements making their mark in this domain, real-time face recognition stands out as a vital tool, promising enhanced security, faster responses, and more effective crime prevention and investigation. The prospect of identifying individuals swiftly and accurately in the field holds immense potential. However, this promising technology encounters significant hurdles when confronted with the intricate and dynamic realities of law enforcement scenarios.

In law enforcement, the stakes are high, and the margin for error is minimal. A matter of seconds can determine the outcome of an operation, and the effectiveness of a face recognition system can make the critical difference in safeguarding public safety. Yet, the practical application of real-time facial recognition is fraught with complexities that threaten its reliability and efficiency. This survey paper delves into these intricacies, seeking to unravel the challenges inhibiting consistent real-time facial recognition in the context of law enforcement.

Foremost among these challenges are occlusions—those formidable obstacles that obscure pivotal facial details. The prevalence of face masks, sunglasses, and other obstructions in today's world can hinder the visibility of critical facial features, rendering many existing recognition systems ineffective. Law enforcement professionals operating in the field encounter these obstructions regularly, making it imperative to craft solutions that can adapt to such occlusions, ensuring accurate and timely identifications.

Equally significant is the complexity of recognizing multiple individuals within crowded settings—a frequent occurrence in many law enforcement contexts. Densely populated areas, public gatherings, and bustling events present a dynamic environment where identifying and tracking individuals in real-time is a formidable task. Swift and accurate recognition becomes paramount to ensuring public safety and efficient law enforcement operations.

Adding another layer of complexity are dynamic and shifting backgrounds. Law enforcement scenarios are fluid, and environmental elements are constantly changing. These dynamic backgrounds can confound recognition systems, leading to discrepancies in identification accuracy. To address this challenge, strategies must evolve to isolate and focus on the facial features irrespective of the ever-changing surroundings.

This survey paper embarks on a comprehensive journey through the realm of face recognition technology, meticulously examining the diverse algorithms, datasets, and techniques employed in recent research endeavours. By consolidating the current understanding and illuminating existing gaps, our objective is to provide a foundational reference for future research endeavours. We aim to guide the development of optimized real-time face recognition systems explicitly tailored for the unique demands of law enforcement. 

The rest of the survey is structured as follows: Section II of the survey is on Face Detection algorithms. In Section III, face tracking algorithms scenario was seen and the STOA algorithms were discussed. Section IV delves with Face Recognition algorithms. Section V provides an overview of accessible datasets for Face Detection and Recognition. And finally we conclude with Section VI. The following sections of this survey paper will explore each of the challenges—occlusions, recognition in crowds, and dynamic background compensation—in greater detail. 