\section{Conclusion} \label{section:conclusion}

In this survey, we have traversed the multifaceted landscape of facial recognition. We delved deep into the core challenges posed by occlusions, crowd dynamics, and ever-shifting backgrounds, each of which has the potential to compromise recognition accuracy.

Our exploration spanned from the foundational algorithms of face detection and recognition to the latest advancements tailored to address the nuances of real-world scenarios. We analysed various methods, ranging from classic computer vision techniques to the pinnacle of deep learning architectures, shedding light on their strengths, weaknesses, and applicability. The role of datasets, as a bedrock for training, validating, and benchmarking these algorithms, was emphasized, emphasizing their significance in shaping the field's trajectory.

Yet, as with any evolving technology, the journey of refining and perfecting real-time face recognition is ongoing. The dynamic interplay of technical challenges, evolving societal contexts (like the widespread adoption of masks), and the ever-present imperative of ethical considerations ensures that this domain will remain at the forefront of research for years to come.

This survey, while comprehensive, is but a snapshot of the current state of affairs. As the field progresses, new challenges will emerge, and novel solutions will be conceived. Nevertheless, our hope is that this work serves as a robust foundation and reference point, catalysing further innovations and guiding researchers and practitioners alike in their endeavours to harness the full potential of real-time face recognition, especially in the service of public safety and law enforcement.