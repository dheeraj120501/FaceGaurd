\section{Datasets} \label{section:datasets}

Datasets play a crucial role in the development, training, and evaluation of face detection and recognition systems. They provide researchers with rich resources such as transfer learning where large face datasets can be used to pre-train models, which can then be fine-tuned on smaller, task-specific datasets is often results in better performance than training on the smaller dataset alone.

Requirements of a good dataset are as follows:

\begin{enumerate}
    \item \textbf{Diversity}: The dataset should encompass a wide variety of facial features, expressions, angles, and occlusions. This ensures that the trained models generalize well to real-world scenarios.

    \item \textbf{High Quality Images}: Images should be of high resolution and clarity. Blurry or low-quality images can hinder the performance of detection or recognition systems.

    \item \textbf{Varied Lighting Conditions}: It should include faces under different lighting conditions - from well-lit to poorly lit scenarios - to challenge and enhance the robustness of algorithms.

    \item \textbf{Age Variability}: Faces from different age groups, ranging from infants to the elderly, should be represented to ensure age-invariance in recognition.

    \item \textbf{Demographic Diversity}: A balanced representation of various ethnicities, genders, and backgrounds is essential to prevent biases in the resulting models.

    \item \textbf{Annotations}: Precise annotations, including bounding boxes for detection and identity labels for recognition, are crucial.

    \item \textbf{Temporal Data}: For video-based recognition systems, the dataset should include video sequences to account for temporal variations and movements.

    \item \textbf{Real-world Scenarios}: Inclusion of "in-the-wild" images or videos where faces are naturally occluded, or in varied expressions and postures, simulates real-world challenges.

    \item \textbf{Scalability}: A good dataset should be large enough to train deep learning models, which often require vast amounts of data.

    \item \textbf{Consent and Ethics}: All data should be collected with proper consent, ensuring the privacy and rights of the individuals are respected.
\end{enumerate}



WebFace260M benchmark \cite{zhu_webface260m_2023}, an ultra-large-scale dataset comprising 4 million identities and 260 million faces was introduced to bridge the data gap between academia and industry. The dataset was still refined by employing the Cleaning Automatically by Self-Training (CAST) pipeline,into WebFace42M, with 2 million identities and 42 million faces. The benchmark and cleaned dataset facilitate efficient model training, resulting in improved face recognition performance and potential solutions for bias mitigation and privacy concerns in the field. The benchmark and cleaned dataset facilitate efficient model training, resulting in improved face recognition performance and potential solutions for bias mitigation and privacy concerns in the field.

Three different kinds of masked face datasets are proposed by Zhongyuan Wang et al. \cite{wang_masked_2023}: the Masked Face Detection Dataset (MFDD), the Real-world Masked Face Recognition Dataset (RMFRD), and the Simulated Masked Face Recognition Dataset (SMFRD). The authors assert that the Real-world Masked Face Recognition Dataset (RMFRD) is the biggest real-world masked face dataset available. Despite this, the sources given do not go into depth on the precise techniques utilised to create the datasets. The DVG-Face framework \cite{fu_dvg-face_2022} was assessed using seven difficult datasets from five HFR tasks: NIR-VIS, Sketch-Photo, Profile-Frontal Photo, Thermal-VIS, and ID-Camera. It outperforms state-of-the-art techniques and produces better results. In order to support research on age-invariant face recognition for their AIM model, Zhao et al. \cite{zhao_towards_2022} assembled a new, extensive Cross-Age Face Recognition (CAFR) benchmark dataset. In-depth tests are carried out on the CAFR dataset as well as additional cross-age datasets (MORPH, CACD, FG-NET) to show how much better the suggested AIM model is than current methods. To confirm that the AIM model can generalise to face recognition in the wild, it is further tested on unconstrained face recognition datasets (YTF, IJB-C).

Authors in \cite{abuzneid_enhanced_2018} used 3 datasets YALE and AT\&T for testing the proposed framework and evaluated their method on the LFW dataset, which is a state-of-the-art benchmark dataset for face recognition.

When the model is used in an uncontrolled environment with noise, occlusion, external lighting, cosmetics, etc.—sometimes referred to as "in the wild"—accuracy decreases. Detection in a controlled setting frequently yields excellent results. However, these sorts of photos are present in datasets. A dataset of this kind is the WIDER FACE dataset, which is used in \cite{putro_high_2021} as a training dataset for the CNN-based lightweight detector. Of the 32,203 photos in it, 12,800 were utilised especially to train the detector. In order to minimise overfitting and enhance the training data, augmentation methods are used. The second dataset has 851 photos with 1335 tagged faces and is called the PASCAL face dataset \cite{putro_high_2021}. This dataset, which is an indoor and outdoor subset of the PASCAL VOC dataset, including changes in backdrop and stance.

Datasets like VGGFace2 and MS-Celeb-1M, which primarily contain data from young, facially beautiful celebrities with makeup, are biased in terms of age and facial appearance, leading to potential performance issues when using pretrained models from these datasets on different audiences \cite{wanyonyi_open-source_2022}.

Tables \ref{det-ds} and \ref{rec-ds} contains a list of popular publicly available datasets found in the literature survey:

\begin{table}[htbp]
\caption{Detection Datasets}
\begin{center}
\begin{tabularx}{\columnwidth}{|X|c|X|}
\hline
\textbf{Dataset} & \textbf{Year}& \textbf{Size} \\
\hline
VGGFace2 \cite{kim_face_2022} & 2018 & 3.31 million images of 9,131 subjects \\
\hline
WIDER Face \cite{kim_face_2022} & 2016 & 32,203 images with 393,703 annotated face bounding boxes \\
\hline
VGGFace2 \cite{kim_face_2022} & 2018 & 3.31 million images of 9,131 subjects \\
\hline
PASCAL Face \cite{feng_detect_2022} & 2012 & 1,335 faces from 851 images \\
\hline
\end{tabularx}
\label{det-ds}
\end{center}
\end{table}
    
    
\begin{table}[htbp]
\caption{Recognition Datasets}
\begin{center}
\begin{tabularx}{\columnwidth}{|X|c|X|}
\hline
\textbf{Dataset} & \textbf{Year}& \textbf{Size} \\
\hline
MS-Celeb-1M \cite{wanyonyi_open-source_2022} & 2016 & 10 million images of 100,000 celebrities \\
\hline
LFW \cite{kim_face_2022} & 2007 & 13,000 images of 5,749 distinct individuals \\
\hline
CASIA NIR-VIS 2.0 \cite{liu_heterogeneous_2022} & 2007 & 17,580 images from 725 subjects \\
\hline
ChokePoint \cite{barquero_rank-based_2021} & 2011 & 54 video sequences captured from 6 different camera views \\
\hline
\end{tabularx}
\label{rec-ds}
\end{center}
\end{table}